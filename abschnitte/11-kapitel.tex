\chapter{Theoretische Grundlagen}
\label{chap:theoretische_grundlagen}

In diesem Kapitel werden die theoretischen Grundlagen dargestellt, die für das Verständnis und die Bearbeitung der Forschungsfrage relevant sind. Zunächst werden zentrale Begriffe definiert, anschließend werden die wichtigsten theoretischen Konzepte erläutert.

\section{Definitionen zentraler Begriffe}
\label{sec:definitionen}

Für das Verständnis der vorliegenden Arbeit ist es notwendig, zunächst die zentralen Begriffe zu definieren und voneinander abzugrenzen.

\subsection{Begriff 1}
\label{subsec:begriff1}

Der Begriff [Begriff 1] wird in der Fachliteratur unterschiedlich definiert. Nach [Autor] handelt es sich dabei um „[wörtliches Zitat]"\footcite[S. XX]{Quelle1}. [Autor 2] hingegen definiert den Begriff als „[wörtliches Zitat]"\footcite[S. XX]{Quelle2}. Für die vorliegende Arbeit wird die Definition von [Autor] verwendet, da [Begründung für die Wahl dieser Definition].

\subsection{Begriff 2}
\label{subsec:begriff2}

Unter [Begriff 2] versteht man [Definition]\footcite[S. XX]{Quelle3}. Diese Definition hat sich in der Fachliteratur weitgehend durchgesetzt und wird daher auch in dieser Arbeit verwendet.

\section{Theoretischer Ansatz 1}
\label{sec:theoretischer_ansatz1}

Der [Theoretischer Ansatz 1] wurde von [Autor] entwickelt und basiert auf der Annahme, dass [grundlegende Annahme]\footcite[S. XX]{Quelle4}. Im Folgenden werden die zentralen Elemente dieses Ansatzes dargestellt.

\subsection{Kernelemente des Ansatzes}
\label{subsec:kernelemente}

Zu den Kernelementen des [Theoretischer Ansatz 1] zählen:

\begin{itemize}
    \item Element 1: [Erläuterung]
    \item Element 2: [Erläuterung]
    \item Element 3: [Erläuterung]
\end{itemize}

\subsection{Kritische Würdigung des Ansatzes}
\label{subsec:kritische_wuerdigung}

Der [Theoretischer Ansatz 1] hat in der Fachliteratur sowohl Zustimmung als auch Kritik erfahren. [Autor] hebt besonders [Stärke des Ansatzes] hervor\footcite[S. XX]{Quelle5}. Kritisch wird hingegen [Schwäche des Ansatzes] gesehen\footcite[S. XX]{Quelle6}. Für die vorliegende Arbeit ist der Ansatz dennoch relevant, da [Begründung für die Verwendung des Ansatzes].

\section{Theoretischer Ansatz 2}
\label{sec:theoretischer_ansatz2}

Als Ergänzung zum [Theoretischer Ansatz 1] wird in dieser Arbeit auch der [Theoretischer Ansatz 2] herangezogen. Dieser wurde von [Autor] entwickelt und fokussiert auf [Fokus des Ansatzes]\footcite[S. XX]{Quelle7}.

\subsection{Grundannahmen}
\label{subsec:grundannahmen}

Der [Theoretischer Ansatz 2] basiert auf folgenden Grundannahmen:

\begin{itemize}
    \item Annahme 1: [Erläuterung]
    \item Annahme 2: [Erläuterung]
    \item Annahme 3: [Erläuterung]
\end{itemize}

\subsection{Anwendbarkeit auf die Forschungsfrage}
\label{subsec:anwendbarkeit}

Der [Theoretischer Ansatz 2] eignet sich besonders für die Bearbeitung der Forschungsfrage dieser Arbeit, da [Begründung]. Die Kombination mit dem [Theoretischer Ansatz 1] ermöglicht eine umfassende theoretische Fundierung der Untersuchung.

\section{Aktueller Forschungsstand}
\label{sec:forschungsstand}

Nachdem die theoretischen Grundlagen dargestellt wurden, wird im Folgenden der aktuelle Forschungsstand zum Thema [Thema] zusammengefasst.

\subsection{Bisherige empirische Untersuchungen}
\label{subsec:bisherige_untersuchungen}

Zum Thema [Thema] wurden bereits verschiedene empirische Untersuchungen durchgeführt. [Autor] untersuchte [Untersuchungsgegenstand] und kam zu dem Ergebnis, dass [Ergebnis]\footcite[S. XX]{Quelle8}. Eine weitere Studie von [Autor] zeigte, dass [Ergebnis]\footcite[S. XX]{Quelle9}.

\subsection{Identifizierte Forschungslücken}
\label{subsec:forschungsluecken}

Trotz der umfangreichen Forschung zum Thema [Thema] bestehen weiterhin Forschungslücken. Insbesondere fehlt es an [konkrete Forschungslücke], die in dieser Arbeit adressiert werden soll. Zudem wurde bisher [weitere Forschungslücke] nicht ausreichend untersucht.

\section{Zusammenfassung und Ableitung des Untersuchungsmodells}
\label{sec:zusammenfassung_modell}

Basierend auf den dargestellten theoretischen Grundlagen und dem aktuellen Forschungsstand wird für die vorliegende Arbeit folgendes Untersuchungsmodell abgeleitet:

[Beschreibung des Untersuchungsmodells]

Dieses Modell dient als Grundlage für die weitere empirische Untersuchung und ermöglicht eine strukturierte Bearbeitung der Forschungsfrage.